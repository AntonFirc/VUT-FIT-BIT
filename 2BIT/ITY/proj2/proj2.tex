\documentclass[a4paper, 11pt]{article}

\usepackage[utf8]{inputenc}
\usepackage{times}
\usepackage[czech]{babel}
\usepackage{amssymb}
\usepackage{amsmath}
\usepackage{amsthm}
\usepackage{setspace}
\setlength{\hoffset}{-2.7cm} 
\setlength{\voffset}{-2cm}
\setlength{\textheight}{25cm} 
\setlength{\textwidth}{18cm}

\newtheorem{veta}{Věta}

\theoremstyle{definition}
\newtheorem{definice}{Definice}[section]

\theoremstyle{definition}
\newtheorem{algoritmus}[definice]{Algoritmus}




\begin{document}

\begin{center}
	\pagenumbering{gobble}
	{\setstretch{0}
		\textsc{\Huge Fakulta informačních technologií\\Vysoké učení technické v~Brně}\\
		\vspace{\stretch{0.382}}
		\LARGE Typografie a publikování - 2.projekt\\
		Sazba dokumentů a matematických výrazů\\
		\vspace{\stretch{0.618}} 
	}
\end{center}
{\Large 2017 \hfill Anton Firc}
\newpage

\pagenumbering{arabic}
\twocolumn
\sloppy
 \section*{Úvod} 
V~této úloze si vyzkoušíme sazbu titulní strany, matematických vzorců, prostředí a dalších textových struktur obvyklých pro technicky zaměřené texty například rovnice (\ref{eq1}) nebo definice \ref{def} na straně \pageref{def}.

Na titulní straně je využito sázení nadpisu podle optického středu s~využitím zlatého řezu. Tento postup byl probírán na přednášce.


\section{Matematický text}

Nejprve se podíváme na sázení matematických symbolů a výrazů v~plynulém textu. Pro množinu $V$ označuje card($V$) kardinalitu $V$.
Pro množinu $V$ reprezentuje $V^*$ volný monoid generovaný množinou $V$ s~operací konkatenace.
Prvek identity ve volném monoidu $V^*$ značíme symbolem $\varepsilon$.
Nechť $V^+ =V^* - \{\varepsilon\}$ Algebraicky je tedy $V^+$ volná pologrupa generovaná množinou $V$ s~operací konkatenace.
Konečnou neprázdnou množinu $V$ nazvěme \emph{abeceda}.
Pro $\omega \in V^*$ označuje $|\omega|$ délku řetězce $\omega$. Pro $W \subseteq V$ označuje occur$(\omega,W)$ počet výskytů symbolů z~$W$ v~řetězci $\omega$ a sym$(\omega,i)$ určuje $i$-tý symbol řetězce $\omega$; například sym$(abcd,3)=c$.

Nyní zkusíme sazbu definic a vět s~využitím balíku \texttt{amsthm}.

\begin{definice} \label{def} \emph{Bezkontextová gramatika} je čtveřice $G=(V,T,P,S)$, kde $V$ je totální abeceda, $T \subseteq V$ je abeceda terminálů, $S \in (V - T)$ je startující symbol a $P$ je konečná množina pravidel tvaru $q: A \rightarrow \alpha$, kde $A \in (V-T)$, $\alpha \in V^*$ a $q$ je návěští tohoto pravidla. Nechť $N = V - T$ značí abecedu neterminálů.
Pokud $q: A \rightarrow \alpha \in P$ , $\gamma , \delta \in V^*$, $G$ provádí derivační krok z~$\gamma A\delta$ do $\gamma \alpha \delta$ podle pravidla $q: A \rightarrow \alpha$, symbolicky píšeme 
$\gamma A\delta \Rightarrow \gamma \alpha \delta [q: A \rightarrow \alpha]$ nebo zjednodušeně $\gamma A\delta \Rightarrow \gamma \alpha \delta$ . Standardním způsobem definujeme ${\Rightarrow}^m$, kde $m \geq 0$ . Dále definujeme tranzitivní uzávěr ${\Rightarrow}^+$ a tranzitivně-reflexivní uzávěr ${\Rightarrow}^*$.
\end{definice}

Algoritmus můžeme uvádět podobně jako definice textově, nebo využít pseudokódu vysázeného ve vhodném prostředí (například \texttt{algorithm2e}).

\begin{algoritmus} \textit{Algoritmus pro ověření bezkontextovosti gramatiky. Mějme gramatiku G = (N, T, P, S).
\begin{enumerate}
	\item \label{krok1} Pro každé pravidlo $p \in P$ proveď test, zda $p$ na levé straně obsahuje právě jeden symbol z~$N$ .
	\item Pokud všechna pravidla splňují podmínku z~kroku~\ref{krok1}, tak je gramatika $G$ bezkontextová.
\end{enumerate}	
}
\end{algoritmus}

\begin{definice} Jazyk definovaný gramatikou $G$ definujeme jako $L(G)=\{\omega \in T|S {\Rightarrow}^* \omega \}$.
\end{definice}
\subsection{Podsekce obsahující větu}

\begin{definice} Nechť $L$ je libovolný jazyk. $L$ je \emph{bezkontextový jazyk}, když a jen když $L = L(G)$, kde $G$ je libovolná bezkontextová gramatika.
\end{definice}

\begin{definice} Množinu ${\mathcal{L}}_{CF}=\{L|L\}$ nazýváme \emph{třídou bezkontextových jazyků.} \end{definice}

\begin{veta} \label{veta} Nechť $L_{abc}=\{a^nb^nc^n|n\geq0\}$. Platí, že $L_{abc} \not \in \mathcal{L}_{CF}$. \end{veta}

\begin{proof}Důkaz se provede pomocí Pumping lemma pro bezkontextové jazyky, kdy ukážeme, že není možné, aby platilo, což bude implikovat pravdivost věty \ref{veta} . \end{proof}

\section{Rovnice a odkazy}

Složitější matematické formulace sázíme mimo plynulý text. Lze umístit několik výrazů na jeden řádek, ale pak je třeba tyto vhodně oddělit, například příkazem \verb|\quad|. 

$$\sqrt[x^2]{y^3_0} \quad \mathbb{N} = \{0,1,2,\ldots\} \quad x^{y^y} \not= x^{yy} \quad z_{i_j} \not \equiv z_{ij} $$

V~rovnici (\ref{eq1}) jsou využity tři typy závorek s~různou explicitně definovanou velikostí.

\begin{equation} \label{eq1}
	\bigg \{ \Big [\big (a+b\big )*c\Big ]^d+1\bigg \}=x 	
\end{equation}
$$
\lim_{x\rightarrow \infty} \frac{\sin^2 x + \cos^2 x}{4} = y
$$

V~této větě vidíme, jak vypadá implicitní vysázení limity $\lim_{n \rightarrow \infty} f(n)$ v~normálním odstavci textu. Podobně je to i s~dalšími symboly jako $\sum_1^n$ či $\bigcup_{A \in \beta}$ . V~případě vzorce $\lim\limits_{x \rightarrow 0} {\frac{\sin x}{x}=1}$ jsme si vynutili méně úspornou sazbu příkazem \verb|\limits|.

\begin{equation}
	\int\limits_a^b f(x)dx = -\int\limits_a^b f(x)dx
\end{equation}
\begin{equation}
	(\sqrt[5]{x^4})' = (x^(\frac{4}{5}))' = \frac{4}{5}x^(-\frac{1}{5}) = \frac{4}{5\sqrt[5]{x}} 
\end{equation}
\begin{equation}
	\overline{\overline{A \vee B}} = \overline{\overline{A} \wedge \overline{B}}
\end{equation}

\section{Matice}

Pro sázení matic se velmi často používá prostředí \texttt{array} a závorky (\verb|\left|,\verb|\right|). 

$$
\left(
\begin{array}{cc}
a+b & b-a \\
\widehat{\xi + \omega} & \widehat{\pi} \\
\mathop{a}\limits^{\rightarrow} & \mathop{AC}\limits^{\longleftrightarrow} \\
0 & \beta
\end{array}
\right)
$$

$$
A = 
\left\| 
\begin{array}{cccc}
a_{11} & a_{12} & \ldots & a_{1n} \\
a_{21} & a_{22} & \ldots & a_{2n} \\
\vdots & \vdots & \ddots & \vdots \\
a_{m1} & a_{m2} & \cdots & a_{mn}
\end{array}
\right\|
$$

$$
\left|
\begin{array}{cc}
t & u \\ v & w
\end{array}
\right|
= tw - uv
$$

Prostředí \texttt{array} lze úspěšně využít i jinde.

$$
\binom{n}{k}
=
\left\{
	\begin{array}{ll}
	\frac{n!}{k!(n-k)!} & \mbox{pro 0} \leq k \leq n \\
	0 & \mbox{pro k} < \mbox{0 nebo}\ k > n
	\end{array}
\right.
$$

\section{Závěrem}

V~případě, že budete potřebovat vyjádřit mate\-matickou konstrukci nebo symbol a nebude se Vám dařit jej nalézt v~samotném \LaTeX u, doporučuji prostudovat možnosti balíku maker \AmS-\LaTeX.
Analogická poučka platí obec\-ně pro jakoukoli konstrukci v~\TeX u.

\end{document}