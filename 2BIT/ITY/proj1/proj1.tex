\documentclass[a4paper, 11pt]{article}

\usepackage[utf8]{inputenc}
\usepackage{times}
\usepackage[czech]{babel}
\setlength{\hoffset}{-2.15cm} 
\setlength{\voffset}{-2cm}
\setlength{\textheight}{24cm} 
\setlength{\textwidth}{17cm}

\title{Typografie a publikování \\ 1. projekt}
\author{Anton Firc\\xfirca00@sud.fit.vutbr.cz}
\date{}


\begin{document}
\sloppy
\twocolumn[\maketitle]
\section{Hladká sazba}
\noindent Hladká sazba je sazba z jednoho stupně, druhu a řezu pí­sma sázená na stanovenou šířku odstavce. Skládá se z odstavců, které obvykle začínají­ zarážkou, ale mohou být sázeny i bez zarážky - rozhodují­cí­ je celková grafická úprava. Odstavce jsou ukončeny východovou řádkou. Věty nesmějí začínat číslicí.

Barevné zvýraznění­, podtrhávání­ slov či různé velikosti písma vybraných slov se zde také nepoužívá. Hladká sazba je určena především pro delší­ texty, jako je napří­klad beletrie. Porušení­ konzistence sazby působí v~textu rušivě a~unavuje čtenářův zrak.

\section{Smíšená sazba} 

\noindent Smíšená sazba má o něco volnější­ pravidla než hladká sazba. Nejčastěji se klasická hladká sazba doplňuje o další řezy pí­sma pro zvýraznění­ důležitých pojmů.\\ Existuje \uv{pravidlo}:

\vspace{4mm}
\parbox{0.81\linewidth}{
    \hspace{4mm} Čí­m ví­ce {\bf druhů, \itshape řezů,}  {\scriptsize velikostí}, barev pí­sma a jiných efektů použijeme, tí­m {\itshape profesionálněji} bude  dokument vypadat. Čtenář tím bude vždy {\Huge nadšen!}}
\vspace{4mm}

{\scshape Tí­mto pravidlem se \underline{nikdy} nesmí­te~ří­dit.} Příliš časté zvýrazňování textových elementů a změny velikosti {\tiny písma} jsou {\huge známkou \bf amatérismu} autora a působí­ {\bf velmi \it rušivě.} Dobře navržený dokument nemá obsahovat ví­ce než 4 řezy či druhy pí­sma. {\ttfamily Dobře navržený dokument je decent\-ní, ne chaotický.}

Důležitým znakem správně vysázeného dokumentu je konzistentní použí­vání­ různých druhů zvýraznění­. To napří­klad může znamenat, že {\bf tučný řez} pí­sma bude vyhrazen pouze pro klíčová slova, {\slshape skloněný řez} pouze pro doposud neznámé pojmy a nebude se to míchat. Skloněný řez nepůsobí­ tak rušivě a použí­vá se častěji. V \LaTeX u jej sází­me raději příkazem\verb|\emph{text}| než \verb|\textit{text}|.


Smíšená sazba se nejčastěji používá pro sazbu vě\-deckých článků a technických zpráv. U delší­ch doku\-mentů vědeckého či technického charakteru je zvykem upozornit čtenáře na význam různých typů zvýrazně\-ní­ v úvodní­ kapitole.

\section{České odlišnosti}

Česká sazba se oproti okolní­mu světu v některých as\-pektech mí­rně liší­. Jednou z odlišností je sazba uvozovek. Uvozovky se v češtině použí­vají­ převážně pro zobrazení­ pří­mé řeči. V menší­ míře se použí­vají­ také pro zvýraznění­ přezdí­vek a ironie. V češtině se použí­vá tento \uv{\bf typ uvozovek} namí­sto anglických "uvozovek". Lze je sázet připravenými příkazy nebo při použití UTF-8 kódování i přímo.

Ve smíšené sazbě se řez uvozovek ří­dí­ řezem první­ho uvozovaného slova. Pokud je uvozována celá věta, sází­ se koncová tečka před uvozovku, pokud se uvozuje slovo nebo část věty, sází­ se tečka za uvozovku.

Druhou odlišností je pravidlo pro sázení­ konců řád\-ků. V české sazbě by řádek neměl končit osamocenou jednopí­smennou předložkou nebo spojkou. Spojkou \uv{a} končit může při sazbě do 25 liter. Abychom \LaTeX u zabránili v sázení­ osamocených předložek, vkládáme mezi předložku a slovo {\bf nezlomitelnou mezeru} znakem\, \~  \, (vlnka, tilda). Pro automatické do\-plnění vlnek slouží­ volně šiřitelný program {\it vlna} od pana Olšáka\footnote{Viz http://petr.olsak.net/ftp/olsak/vlna/.}.
\end{document}